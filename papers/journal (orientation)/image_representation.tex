\chapter{Techniques}

\section{Image Representation}
\label{s:filtering}
Our first task is to reduce an image region that contains redundant data in a high dimensional space to a feature vector that encapsulates only the most relevant details in a low dimensional space.



Early attempts to detect linear structure in an image (such as the Line Operator, or `LinOp'~\cite{Dixon_Taylor_IPC79}) were based on modifications of template matching that could be na\"ively applied to mammograms~\cite{Parr_etal_SPIE97}.

Second derivatives of the mammographic image surface~\cite{Cerneaz_Brady_CVVRRM95}. 

Second order Gaussian derivatives~\cite{Karssemeijer_teBrake_TMI96}. 

Zwiggelaar et al compared all these methods using simulated mammographic images and reported that Linop gave the best results in terms of area under the receiver operating characteristic (ROC) curve~\cite{Zwiggelaar_etal_TMI04}. 

More recently, researchers have recognised the importance of using local phase information, particularly in distinguishing between strong edges and genuine curvilinear structure. Rangayyan and Ayres described an approach based on Gabor filters~\cite{Rangayyan_Ayres_MBEC06}. 

Schenk and Brady~\cite{Schenk_Brady_IWDM02} and McLoughlin et al~\cite{McLoughlin_etal_SPIE02} used sets of steerable complex filters~\cite{Freeman_Adelson_TPAMI91} at multiple scales to compute local energy, orientation and phase at each pixel. 

Wai et al~\cite{Wai_etal_MICCAI04} used the monogenic signal as more efficient way of calculating local phase and orientation at multiple scales, detecting curvilinear structure using amplitude-weighted local phase congruency. 

Overall, the conclusion that can be drawn from the literature is that the detection of curvilinear structure benefits from access to local phase and magnitude information at multiple scales.

DT-CWT/RF: the method described in this paper.
Monogenic: the monogenic-signal phase congruency approach~\cite{Wai_etal_MICCAI04}.
Linop: the Line operator~\cite{Dixon_Taylor_IPC79,Parr_etal_SPIE97}.
Gaussian: the directional Gaussian 2nd derivatives~\cite{Karssemeijer_teBrake_TMI96}.


\subsection{First derivatives}
A simple but na\"ive filtering approach uses smoothed derivatives, $\Gx$ and $\Gy$ (\fref{f:filters}a-b), and their corresponding responses, $\Ix$ and $\Iy$, to compute the direction in which gradient is strongest; the perpendicular to this gradient defines the line orientation.

Defining $\G_{(1)}(\theta)$ as the first derivative filter at angle $\theta$,
%
\begin{align}
G_{(1)}(\theta)
	&= 	\frac{\partial G}{\partial r} \\
	&= 	\frac{\partial G}{\partial x}\frac{\partial x}{\partial r} +
			\frac{\partial G}{\partial y}\frac{\partial y}{\partial r} \\
	&= 	\Gx \cos(\theta) + \Gy \sin(\theta)
\label{e:dG}
\end{align}

\noindent where $\Gx = G_{(1)}(0^\circ)$ and $\Gy = G_{(1)}(90^\circ)$. The response to this filter is
%
\begin{align}
R_{(1)}(\theta)
	&= 	\G_{(1)}(\theta) \ast I \\
	&=	(\Gx \cos(\theta) + \Gy \sin(\theta)) \ast I \\
	&=	(\Gx \ast I) \cos(\theta) + (\Gy \ast I) \sin(\theta) \\
	&=	\Ix \cos(\theta) + \Iy \sin(\theta)
\label{e:R1}
\end{align}

\noindent where $\Ix$ and $\Iy$ are the responses to $\Gx$ and $\Gy$, respectively. That the response at any $\theta$ is a linear sum of the response at two orientations is the property of \emph{steerability} that makes computing the response at any angle efficient. Differentiating and equating to zero gives
%
\begin{align}
\frac{d}{d\theta}R_{(1)}
	&= -\Ix \sin(\theta) + \Iy \cos(\theta) = 0 \\
\Rightarrow \tan(\theta)
	&= \frac{\Iy}{\Ix}.
\label{e:t1}
\end{align}

One problem with this approach is that although the gradient has a clear direction (\eg~dark to light) its perpendicular does not: there are two opposite directions, both with zero gradient, and the estimated orientation is arbitrary up to a rotation of $180^\circ$. Though technically we do not distinguish between opposite orientations, opposites cancel when computing statistics (\eg~the mean orientation over a local patch). 


\subsection{Squared responses}
This problem can be avoided by considering the squared response to $G_{(1)}(\theta)$:

\begin{align}
R_{(1)}^2(\theta)
	&=	(\Ix \cos(\theta) + \Iy \sin(\theta))^2 \\
	&= 	\Ix^2 \cos^2(\theta)+\Iy^2 \sin^2(\theta)+2\Ix\Iy\sin(\theta)\cos(\theta) \\
	&= 	\Ix^2 \cos^2(\theta)+\Iy^2 \sin^2(\theta)+\Ix\Iy\sin(2\theta)
\label{e:R1sqr}
\end{align}

\noindent such that differentiating and equating to zero gives
%
\begin{align}
\frac{d}{d\theta}R_{(1)}^2
	&= 	-2\Ix^2 \cos(\theta)\sin(\theta) + 2\Iy^2 \sin(\theta)\cos(\theta) + 2\Ix\Iy\cos(2\theta) \\
	&= 	(\Iy^2-\Ix^2) \sin(2\theta) + 2\Ix\Iy\cos(2\theta) = 0 \\
\Rightarrow \tan(2\theta)
	&= 	\frac{2\Ix\Iy}{\Ix^2-\Iy^2}.
\label{e:t1sqr}
\end{align}

\noindent where, by doubling the angle, opposite orientations reinforce each other rather than cancel out~\cite{Mardia_Jupp_00}. 


%\subsection{Squared filters}
%\begin{align}
%\tan(2\theta)
%	&= 	\frac{2(\Gx\Gy \ast I)}{(\Gx^2 \ast I)-(\Gy^2 \ast I)}.
%\label{e:t1fsqr}
%\end{align}
%
%\noindent with `cloverleaf' type filters.


\subsection{Second derivatives}
A further criticism of the first derivative approach is that $\Gx$ and $\Gy$ respond most strongly at the edges (rather than the centre) of a bar. In fact, any approach based on odd image filters (including the monogenic signal~\cite{Felsberg_Sommer_TSP01}) fails at the centre of symmetric bar features where both $\Ix$ and $\Iy$ are close to zero such that line orientation is poorly defined. Taking products of the responses does not help in this respect -- the products are also close to zero -- but filters based on second derivatives (and that are therefore even) can provide a more stable solution.

We define $\G_{(2)}(\theta)$ as the second derivative filter at angle $\theta$ such that

\begin{align}
\G_{(2)}(\theta)
	&= 	\frac{\partial}{\partial x}(\Gx \cos(\theta) + \Gy \sin(\theta))\frac{\partial x}{\partial r} \notag\\
		&\qquad + \frac{\partial}{\partial y}(\Gx \cos(\theta) + \Gy \sin(\theta))\frac{\partial y}{\partial r} \\
%
	&= 	(\Gxx \cos(\theta) + \Gyx \sin(\theta))\cos(\theta) \notag\\
		&\qquad + (\Gxy \cos(\theta) + \Gyy \sin(\theta))\sin(\theta) \\
%
	&= 	\Gxx\cos^2(\theta) + \Gyy\sin^2(\theta) + 2\Gxy\sin(\theta)\cos(\theta) \\
%
	&=	\Gxx\cos^2(\theta) + \Gyy\sin^2(\theta) + \Gxy\sin(2\theta)
\label{e:ddG}
\end{align}

As noted some years ago (and exploited in mammography applications~\cite{Karssemeijer_teBrake_TMI96}), second derivatives are also steerable and are therefore highly efficient~\cite{Freeman_Adelson_TPAMI91,Koenderink_vanDoorn_TPAMI92}. For further efficiency, we differentiating $\Gx$ and $\Gy$ to get the equivalent \emph{separable} filters $\Gxx$, $\Gyy$ and $\Gxy$ (\fref{f:filters}c-e), and use these to compute the response to $G_{(2)}(\theta)$,

\begin{align}
R_{(2)}(\theta)
	&= 	\G_{(2)}(\theta) \ast I \\
	&=	\Ixx\cos^2(\theta) + \Iyy\sin^2(\theta) + \Ixy\sin(2\theta).
\label{e:R2}
\end{align}

\noindent where $\Ixx$, $\Iyy$ and $\Ixy$ are the responses to $\Gxx$, $\Gyy$ and $\Gxy$, respectively. If we differentiate with respect to $\theta$, we find a stationary point at

\begin{align}
\frac{d}{d\theta}R_{(2)}
	&= 	-2\Ixx\cos(\theta)\sin(\theta) + 2\Iyy\sin(\theta)\cos(\theta) + 2\Ixy\cos(2\theta) \\
	&= 	(\Iyy-\Ixx)\sin(2\theta) + 2\Ixy\cos(2\theta) = 0 \\
\Rightarrow \tan(2\theta)
	&= 	\frac{2\Ixy}{\Ixx-\Iyy}.
\label{e:t2}
\end{align}

Since \eref{e:t1sqr} and \eref{e:t2} actually give values of $2\theta \pm k\pi$, solving for $\theta$ gives two orientations (one minimum and one maximum) that are $90^\circ$ apart. An analytic solution then requires us to compute the actual response at the two solutions in order to estimate line orientation. 


\subsection{Haar-like Approximations}
At this point it is helpful to examine the equivalent filters involved: $\Ixy$ and $\Ixx-\Iyy$ (\fref{f:filters}e,f). In particular, we note that their distinctive `cloverleaf' appearance is well approximated by highly efficient Haar-like features (\fref{f:filters}g,h) that have demonstrated success since their introduction for face detection~\cite{Viola_Jones_IJCV04}. To approximate $\Gxx-\Gyy$, we used a modification of the summed area table to compute features at $45^\circ$~\cite{Lienhart_Maydt_ICIP02}.


\subsection{The Dual Tree Complex Wavelet Transform (\dtcwt)}

\begin{figure}[t]
\centering
\begin{tabular}{c c c c}
\includegraphics[width=0.15\columnwidth]{\figpath/Gx} &
\includegraphics[width=0.15\columnwidth]{\figpath/Gy} &
\includegraphics[width=0.15\columnwidth]{\figpath/Gxx} &
\includegraphics[width=0.15\columnwidth]{\figpath/Gyy} \\
(a) & (b) & (c) & (d) \\
\includegraphics[width=0.15\columnwidth]{\figpath/Gxy} &
\includegraphics[width=0.15\columnwidth]{\figpath/Gxx-Gyy} &
\includegraphics[width=0.15\columnwidth]{\figpath/haar_sin} &
\includegraphics[width=0.15\columnwidth]{\figpath/haar_cos} \\
(e) & (f) & (g) & (h) \\
\end{tabular}
%
\caption{(a,b)~First derivatives, $\Gx$ and $\Gy$; (c-e)~Second derivatives, $\Gxx$, $\Gyy$ and $\Gxy$; (f)~$\Gxx-\Gyy$; (g,h)~Haar-like approximations to $\Gxy$ and $\Gxx-\Gyy$.}
\label{f:filters}
\end{figure}

Although filters based on second derivatives can give accurate output when positioned at the centre of a line feature, accuracy when away from the centre of the line cannot be assured. It has been noted, therefore, that accommodating variation in \emph{phase} of the filter can offer benefits in this regard. Though Gabor filters have historically been a popular choice~\cite{Daugman_TASSP88}, we use a more recent contender known as the Dual-Tree Complex Wavelet Transform (\dtcwt~\cite{Kingsbury_PTRSLA99}). 


\subsection{The Dual-Tree Complex Wavelet Transform}
Wavelet transforms have been used extensively in image processing and analysis to provide a rich description of local structure. The dual-tree complex wavelet transform has particular advantages because it provides a directionally selective representation with approximately shift-invariant coefficient magnitudes and local phase information~\cite{Kingsbury_ACHA01}. The DT-CWT combines the outputs of two discrete transforms, using real wavelets differing in phase by 90\deg, to form the real and imaginary parts of complex coefficients. For 2-D images, it produces six directional sub-bands, oriented at $\pm$15\deg, $\pm$45\deg and $\pm$75\deg, at each of a series of scales separated by factors of two. The DT-CWT is less expensive to compute and provides a richer description (magnitude and phase at six orientations rather than one) than the monogenic signal used by Wai et al~\cite{Wai_etal_MICCAI04}.


We have argued for the rich description of local structure provided by the DT-CWT. We now have to decide how to use this information to compute a single measure of curvilinear structure probability. One solution would be to select the maximum of the six oriented sub-band coefficients at each scale, and combine them across scale in a measure of phase congruency, as in the monogenic signal based method of Wai et al~\cite{Wai_etal_MICCAI04}, but this would discard potentially useful information. Instead, we pose the problem as a classification task in which we attempt to learn the patterns of DT-CWT coefficients associated with pixels belonging to either linear structures or background. We construct a feature vector characterising each pixel, by sampling DT-CWT coefficients (in phase/magnitude form) from the six oriented sub-bands in each of the s finest decomposition scales from a neighbourhood centred on the pixel. This involves interpolating the coefficients obtained at coarse scales to provide a response at every pixel using the band-pass interpolation method of Anderson et al~\cite{Anderson_etal_ICIP05}. To improve the rotational symmetry of the coefficients, we apply the transformations recommended by Kingsbury~\cite{Kingsbury_ECSP06}. That is, an additional set of filters are used to reduce the wave frequencies of the 45\deg and 135\deg sub-bands so that they lie closer to the 15\deg, 75\deg, 105\deg and 165\deg bands. The six sub-bands are then multiplied by {i, -i, i, -1, 1, -1} respectively, so that the phase at the centre of the impulse response of each wavelet is zero. Finally, to achieve 180\deg rotational symmetry, we replace any coefficient with negative imaginary part with its complex conjugate.


The \dtcwt~has advantages over similar methods in that it combines the computational efficiency of decimation (\ie~multiresolution filtering is achieved by successively down-sampling the image by factors rather than by increasing filter size) while producing approximately shift-invariant coefficients. \dtcwt~coefficients at six directional sub-bands -- oriented at approximately $\pm15^\circ$, $\pm45^\circ$ and $\pm75^\circ$ within each scale -- can be interpolated to obtain a representation of local energy and phase and any spatial location. To achieve $180^\circ$ rotational symmetry, we replace any coefficient that has a negative imaginary part with its complex conjugate (equivalent to taking the absolute value of the local phase) and adjust the six sub-bands such that the phase at the centre of the impulse response of each wavelet is zero~\cite{Kingsbury_ECSP06,Berks_etal_IPMI11}.

When dealing with a complex response, $c$, we separate its magnitude, $|c|$, from its phase, $\angle c$. Since orientation is only defined up to a rotation of $180^\circ$, however, a point with phase $\phi$ displaced by $d$ from the centre of a line is indistinguishable from a point with phase $-\phi$ displaced by $-d$ from the same line when looking in the opposite direction; we therefore take the absolute value of phase, $|\angle c|$, at each pixel.


\subsection{Multiresolution Filtering}
Since curvilinear structure appears at a number of scales in the image (\eg~from fine spicules to thick ducts in mammograms), it is also important to filter the image at several scales to capture all structure~\cite{Lindeberg_IJCV98b}. Having applied our filter bank at a number of scales, an important question is how to interpret the responses at each scale. One option is to discard all scales but that with the strongest overall response~\cite{Karssemeijer_teBrake_TMI96} and use the responses from the discrete orientations at the selected scale to determine orientation analytically. In this work, however, we investigate the alternative approach whereby we use the responses at all scales as input to a regressor that predicts the orientation directly. This general purpose approach has the added advantage that it can be applied for filter banks where an analytic solution is not obvious (such as the \dtcwt).