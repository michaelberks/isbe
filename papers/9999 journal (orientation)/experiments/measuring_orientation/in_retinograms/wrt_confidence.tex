We also investigated the variation of mean angular error with respect to the magnitude of the orientation vector predicted by the regression forests (\fref{f:retina_graphs}d). The consistent reduction in angular error with increasing magnitude for all feature types confirms that this is indeed a measure of confidence in the forest prediction, whilst the same cannot be said of the magnitude of the analytic response.

Moreover, careful examination shows that while orientation magnitude is generally high at vessels (as a classifier it achieves $A_z = 0.901$) it is lower at bifurcations and crossings where orientation is not well defined, and thus provides information in addition to the classification probabilities. 

For example, where angular measurements are of direct interest (\eg~vessel tortuosity~\cite{Hart_etal_IJMI99}) the segmentation may suggest which pixels to include in the analysis with orientation confidence weighting the inputs. Thus the orientation confidence could sensibly down-weight the contribution of points where orientation is ill-defined even though vessel probability may be high.

Alternatively, in further processing of vessel probabilities such as using a tracking algorithm to group connected pixels, the orientation confidence can be used to control the spread of potential paths, with a narrow focus along vessel centres and a wide spread at bifurcations where the paths necessarily diverge.