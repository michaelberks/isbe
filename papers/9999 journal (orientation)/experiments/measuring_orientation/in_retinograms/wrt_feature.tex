\label{s:exp_retinogram_orientation_wrt_feature}
% Compare all feature types with the most powerful and flexible regressor (e.g. Random Forests) so that complex feature types are not disadvantaged.

Features based on odd filters -- first derivatives of the Gaussian and the monogenic signal, for example -- were consistently outperformed by those that included even filters (\fref{f:retina_graphs}). This can be attributed to the high errors that occur when using only odd filters at the centre of a vessel where there is little or no image gradient; this failure is particularly acute for the most narrow vessels, consisting of a single pixel, where the whole vessel is a centreline by definition. 

Of the two features that use even filtering -- the second Gaussian derivatives and the \dtcwt{} -- the \dtcwt{} provided better estimates under most conditions. This suggests that using both odd and even filters (such that phase information becomes available) does indeed improve performance, though at a cost in computational demand.

%With the exception of the boosted regressor, the Haar-like approximation exhibited similar performance to the second derivative, suggesting that it may be used effectively in scenarios where efficiency is a concern.


