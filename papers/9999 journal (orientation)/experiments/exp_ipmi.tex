We have conducted a systematic evaluation of the performance of our method for curvilinear structure detection and classification, using both synthetic and real mammogram data, comparing seven competing approaches:

\begin{itemize}
\item Monogenic: the monogenic-signal phase congruency approach~\cite{Wai_etal_MICCAI04}.
\item	Monogenic/RF: the raw responses used in Monogenic, combined using RF classification.
\item Linop: the Line operator~\cite{Dixon_Taylor_IPC79,Parr_etal_SPIE97}.
\item	Linop/RF: the raw responses used in Linop, combined using RF classification.
\item	Gaussian: the directional Gaussian 2nd derivatives~\cite{Karssemeijer_teBrake_TMI96}.
\item Gaussian/RF: the raw responses used in Gaussian, combined using RF classification.
\item DT-CWT/RF: the method described in this paper.
\end{itemize}

Monogenic, Linop and Gaussian are representative of the current state of the art in line detection. Monogenic/RF, Linop/RF and Gaussian/RF are natural variants, in which the intermediate multiscale responses previously used to construct the detection outputs are instead combined to given feature representation at each pixel that can subsequently classified using a random forest. 

These learning variants were developed for two reasons. Firstly, when analyzing quantitative results for detection performance, they allow us to decouple the effect of random forest learning from the effect due to the type of representation used. Secondly, unlike their original variants, Monogenic/RF, Linop/RF and Gaussian/RF can be used in the spicule classification experiment described in 5.3.

In what follows, we present both qualitative and quantitative results for detection and classification for each method outlined above. 