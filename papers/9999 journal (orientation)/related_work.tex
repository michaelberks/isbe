\comment{
There is no 3D analogue of the \dtcwt{}. Therefore, we should avoid 3D datasets and related work as much as possible - this paper is solely about the analysis of 2D image structures. This includes aerial photography, retinal images, fingerprints and palmprints, surface inspection, and possibly fibre analysis.
}

Because previous attempts at detecting and measuring curvilinear structures have been surveyed elsewhere, from the general~\cite{Papari_Petkov_IVC11} to the application-specific~\cite{Kirbas_Quek_ACMCS04,Lesage_etal_MIA09}, here we briefly review only a few papers in applications areas of specific interest. We will also discount the very basic detection methods that simply threshold an image~\cite{Jiang_Mojon_TPAMI03} and very complex methods that use techniques such as phase fields~\cite{Peng_etal_IJCV09}, leaving us free to focus on the most popular approaches that apply a bank of filters to the image before interpreting the responses in a way that accentuates the `lineness' at every pixel.

Early examples modelled a curvilinear structure as an image `ridge', typically defined using second order derivatives and formulated either as a Hessian problem~\cite{Frangi_etal_MICCAI98,Sato_etal_MIA98} or as responses to a derivative filter~\cite{Staal_etal_TMI04,Aylward_Bullitt_TMI02,Steger_TPAMI98,Koenderink_vanDoorn_TPAMI92}. Interpreting the responses to these filters used hand-crafted equations based on the differential geometry of the image `surface'~\cite{Frangi_etal_MICCAI98,Sato_etal_MIA98}.

With the maturing of machine learning, it was recognized that hand-crafting could be replaced by flexible statistical models whose parameters were optimized to agree with input-output training example pairs; Gaussian mixture models~\cite{Soares_etal_TMI06}, %
artificial neural networks~\cite{Marin_etal_TMI11,Minh_Hinton_ECCV10}, %
support vector machines~\cite{Ricci_Perfetti_TMI07,Gonzalez_etal_CVPR09}, and %
$k$-nearest neighbour classifiers~\cite{Staal_etal_TMI04} were all popular choices. Not only are these methods well-established and understood within machine learning, but a learnt statistical model can also accommodate different sensing modalities (with different noise properties) and other `stuff' that is hard to hand-craft, making the resulting algorithms more easily transferrable.

In addition, learnt statistical models can be applied to \emph{any} vector of image features to predict the quantity of interest, and are not tied to a particular feature type with specific theoretical properties. Furthermore, flexible statistical models permit the option to include hand-crafted, application-specific features where beneficial~\cite{Staal_etal_TMI04}. As a result, subsequent studies were free to use alternative filter banks and features based on %
matched filters or templates~\cite{Chaudhuri_etal_TMI89,Pechaud_etal_CVPR09,Dixon_Taylor_IPC79,Hoover_etal_TMI00,Ricci_Perfetti_TMI07}; %
derivatives of a first~\cite{Cai_Chung_MICCAI06} or higher than second~\cite{Gonzalez_etal_CVPR09} order; %
Gabor filters~\cite{Soares_etal_TMI06,Dabbah_etal_MIA11}; %
moments~\cite{Marin_etal_TMI11}; %
principal component analysis~\cite{Minh_Hinton_ECCV10}; %
wavelet transforms; %
and the monogenic signal. %
Some of these filter banks have been evaluated previously in a brief comparison~\cite{Ayres_Rangayyan_JEI07}, though the authors admit that not all filters were compared exactly on a like-for-like basis; our work builds on this comparison.

As well as permitting flexibility in the choice of input feature vector, learnt statistical predictors can be used to predict more than just the presence of a line. In recent studies, for example, orientation~\cite{Zwiggelaar_etal_TMI04,Ayres_Rangayyan_JEI07}, width~\cite{Steger_TPAMI98,Zwiggelaar_etal_TMI04} and multi-class labels~\cite{Zwiggelaar_etal_TMI04} have all been predicted from image feature vectors.

Other facets of tube detection and measurement are interesting though not directly relevant to this work: %
detecting lines at multiple scales~\cite{Lindeberg_IJCV98,Sato_etal_MIA98}; %
dealing with junctions and bifurcations where multiple lines meet at a point~\cite{Chen_etal_TPAMI00};
regularizing estimated lines via snakes~\cite{Laptev_etal_MVA00} or dynamic programming~\cite{Gruen}; %
and line following or tracking~\cite{Aylward_Bullitt_TMI02,Perez_etal_ICCV01}.

