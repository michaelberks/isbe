\label{s:dataset_synthmamm}
%
Given the practical difficulties of obtaining a precise binary labelling of curvilinear structure in real mammograms, we generated synthetic images~\cite{Berks_PhD10} with which to train and test algorithms for analysis. Specifically, to create each image we took a $4 \by 4$ cm ($512 \by 512$ pixel) region from a randomly selected real mammogram with 256 grey levels, and filtered out the high frequency linear structures, leaving only the coarse background texture with local image noise across all frequencies. We then superimposed linear structures with randomly chosen parameters (orientation, width, peak intensity and profile shape) onto the background.

\begin{itemize}
\item Orientations were drawn from a uniform distribution over the range [0, $\pi$] radians.

\item Widths were drawn from a uniform distribution over [4, 16] pixels.

\item Peak intensity [1,255] grey-levels (relative to images scaled 0-255) from an exponential distribution with half-width 4 grey-levels;

\item Profile shape $S$ determined by the equation $S = a + (1-a) \sin(x)$ for offsets $x \in (0,\pi)$, where the 'squareness' parameter $a$ determines how close the shape is to a pure half-cycle $\sin$ or a rectangle and is drawn uniformly from [0,1].
\end{itemize}

The result was an image whose statistics were close to those of a real mammogram, but with known line parameters (\fref{f:synthetic_responses}).

In this manner, we generated 10\,460 training images and 4\,903 test image patches using 72 training and 30 test normal\comment{normal?} mammograms.

The literature suggests (and we ourselves observed) that the best performance is obtained by training with a balanced dataset -- in this case with the same number of foreground (curvilinear structure) and background examples. Having created these images, therefore, we sampled feature vectors at pixels on the centrelines of the superimposed linear structure, and an equal number of `background' pixels to serve as positive and negative training sets, respectively.\comment{for detection, of course} 

During training, backgrounds were randomly sampled from the 10\,460 training patches and a single line was added to the centre of the patch. These images were produced 'on-the-fly' during each tree-building step of random forest construction and no permanent set was maintained. 

(We simply generated images until we had the desired 200k points, then built the tree; we did this for 200 trees.)

For testing, 100 backgrounds were randomly selected from the test patches. To each, multiple lines were added sequentially, with the number and position of lines varying randomly.

