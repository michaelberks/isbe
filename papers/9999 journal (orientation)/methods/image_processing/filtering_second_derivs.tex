\label{s:filtering_secondderivs}
%
\begin{figure}[t]
\centering
\begin{tabular}{@{}c c c c@{}} % @{} removes padding around the edge of the table
\includegraphics[width=0.2\columnwidth]{\figpath/filtering/Gxx} &
\includegraphics[width=0.2\columnwidth]{\figpath/filtering/Gyy} &
\includegraphics[width=0.2\columnwidth]{\figpath/filtering/Gxy} &
\includegraphics[width=0.2\columnwidth]{\figpath/filtering/Gxx-Gyy} \\
(a) & (b) & (c) & (d)
\end{tabular}
%
\caption{(Second derivatives (a) $\Gxx$ (b) $\Gyy = \Gxx^T$, (c) $\Gxy$, and (d) $\Gxx-\Gyy$.}
\label{f:filters_secondderivs}
\end{figure}%
%
The directional second-order derivative of a Gaussian generates an even (\ie~symmetric) image filter that resembles a bar or ridge feature. Like their first-order counterparts, second-order derivatives are steerable though they require three rather than two basis filters. The second-order basis filters are not separable, though a different formulation generates three equivalent basis filters -- $\Gxx$, $\Gyy$ and $\Gxy$ (\fref{f:filters_secondderivs}) -- that are. The response to a second-order derivative is given by
%
\begin{equation}
R(\theta) = \Ixx \cos^2(\theta) + \Iyy \sin^2(\theta) + \Ixy \sin(2\theta)
\label{e:secondderivs_response}
\end{equation}
%
\noindent where $\Ixx = \Gxx\ast I$, $\Iyy = \Gyy\ast I$ and $\Ixy = \Gxy\ast I$ are the responses to the three separable filters. This response function has four stationary points in the range $[0,2\pi)$, occuring at
%
\begin{equation}
\theta = \frac{1}{2} \tan^{-1}\left( \frac{2\Ixy}{\Ixx-\Iyy} \right).
\label{e:secondderivs_orientation}
\end{equation}
%
\noindent Two of these points, separated by $\pi\,\rad$ because the filter is rotationally symmetric, correspond to the directions in which the filter is aligned with the feature and the \emph{absolute} value of the response is maximal; the other two points correspond to the two perpendicular directions. 

Because, however, the maximal absolute response may correspond to either a maximum or minimum (depending on whether the underlying feature is light-on-dark or vice versa) the only way to find out which of the two perpendicular directions has maximal absolute value is to evaluate the response at both and choose the direction corresponding to the larger absolute value. As a result, estimating orientation from second-order derivatives becomes a nonlinear problem.

Estimating direction or orientation from second-order derivatives also suffers from an instability when both $\Ixy$ and $\Ixx-\Iyy$ are close to zero. Unfortunately, this happens at edges, whose orientation we may wish to estimate.

Efficient approximations to computing second-order derivatives of the Gaussian can be achieved through Haar-like approximations to the derivative filters~\cite{Bay_etal_CVIU08} or by approximating the Gaussian filter and applying local finite differences~\cite{Kovesi_DICTA10}.