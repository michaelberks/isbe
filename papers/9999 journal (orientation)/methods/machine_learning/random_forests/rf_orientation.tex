% Points specific to estimating orientation with a Random Forest
When using a Random Forest as a regressor to predict orientation, we must take special care to ensure that the sample statistic (\eg~variance) used to partition the training data is appropriate. One statistic that does respect the circular nature of orientation is the angular dispersion, and so it is a natural choice for choosing an optimal partition of the sample.

When using a pruned tree, we can replace the samples at each leaf by a histogram that captures  any multimodal properties of the output (where lines cross, for example). Alternatively, we can replace the samples at each leaf by their summary statistics to save memory. In the case of orientation, the mean of the complex values gives both the average orientation (the angle of the mean vector) and the angular dispersion of the sample (the magnitude of the mean vector). This therefore provides a measure of confidence in the estimate for a single tree (in contrast to when using unpruned trees where every estimate has an identical confidence of 1).

When computing the final output of the forest, we can also take the mean over all $F$ outputs to give an estimate of orientation that is weighted by the confidence in each individual prediction. This output vector will also have a magnitude in the range $[0,1]$ that indicates confidence in the overall prediction produced by the forest.
