\chapter{Classifying Linear Structure}

Classification is of interest when it is important to distinguish between subtly different structures which may be present within the same image - for example, rivers and roads

The learning approach described above can also be used to differentiate between different kinds of curvilinear structure. The hypothesis is that the cross-sectional intensity profiles of structures differ in systematic ways between types of mammographic structure (as suggested by Zwiggelaar et al~\cite{Zwiggelaar_etal_TMI04}), and that profile shape is effectively captured by the DT-CWT coefficients - particularly in the phase components. In the experiments described below we concentrated on the task of distinguishing between spicules, which are a sign of malignancy, and other curvilinear structures. Although it was not realistic to create a training set of mammogram images with all spicules annotated, we were able to obtain expert annotations of a reasonably large number of spicules in a set of mammogram patches containing malignant tumours (see section 5.1 for details). These annotations were used to select pixels for the positive training set. To create a balanced training set we sampled feature vectors from the same number of pixels in a set of normal mammogram patches, such that the distribution of curvilinear structure probabilities was the same as for the spicule training set. Using this balanced training set, we built a random forest classifier to perform spicule/non-spicule classification.

The literature on classifying curvilinear structures in mammograms is much more limited. We are aware of the work of Zwiggelaar et al~\cite{Zwiggelaar_etal_TMI04}, which demonstrated the feasibility of distinguishing between different types of structure using cross-sectional profiles obtained from manually annotated curvilinear structures, but did not obtain very satisfactory results when the method was fully automated.  We recently reported preliminary classification (but not detection) results using our current approach~\cite{Chen_etal_IWDM10}.


\section{Experimental Evaluation}
We have conducted a systematic evaluation of the performance of our method for curvilinear structure detection and classification, using both synthetic and real mammogram data, comparing seven competing approaches:

\begin{itemize}
\item DT-CWT/RF: the method described in this paper.
\item Monogenic: the monogenic-signal phase congruency approach~\cite{Wai_etal_MICCAI04}.
\item Linop: the Line operator~\cite{Dixon_Taylor_IPC79,Parr_etal_SPIE97}.
\item	Gaussian: the directional Gaussian 2nd derivatives~\cite{Karssemeijer_teBrake_TMI96}.
\item	Monogenic/RF: the raw responses used in Monogenic, combined using RF classification.
\item	Linop/RF: the raw responses used in Linop, combined using RF classification.
\item Gaussian/RF: the raw responses used in Gaussian, combined using RF classification.
\end{itemize}

Monogenic, Linop and Gaussian are representative of the current state of the art in line detection. Monogenic/RF, Linop/RF and Gaussian/RF are natural variants, in which the intermediate multiscale responses previously used to construct the detection outputs are instead combined to given feature representation at each pixel that can subsequently classified using a random forest. These learning variants were developed for two reasons: firstly, when analyzing quantitative results for detection performance, they allow us to decouple the effect of random forest learning from the effect due to the type of representation used. Secondly, unlike their original variants, Monogenic/RF, Linop/RF and Gaussian/RF can be used in the spicule classification experiment described in 5.3.

In what follows, we present both qualitative and quantitative results for detection and classification for each method outlined above. 

\subsection{Data}
\paragraph{Synthetic Data.} We used two sets of synthetic images containing curvilinear structures, with known ground truth, to train and test curvilinear structure detection. We randomly extracted 4 \by 4 cm (512 \by 512 pixel) mammographic backgrounds with 256 grey-levels 72 (30) normal mammograms for the training (test) set, resulting in 10460 training patches and 4903 test patches, from which naturally occurring linear structures were filtered out. Lines were added to these backgrounds, with parameters drawn from the following distributions: orientation [0, ?] uniform; width [4, 16] pixels uniform; peak intensity [1,255] grey-levels (relative to images scaled 0 - 255) from an exponential distribution with half-width 4 grey-levels; profile shape ? determined by the equation ? = ? + (1- ?) sinx for offsets x  (0,?), where the 'squareness' parameter ? determines how close the shape is to a pure half-cycle sin or a rectangle and is drawn uniformly from [0,1]. The range of widths, distribution of contrasts (with low contrast lines much more likely than high contrast lines) and variability in profile shape were chosen to mimic what is found in real mammograms.

During training, backgrounds were randomly sampled from the 10460 training patches and a single line was added to the centre of the patch. These images were produced 'on-the-fly' during each tree-building step of random forest construction as described in section 5.2 and no permanent set was maintained.

For testing, 100 backgrounds were randomly selected from the test patches. To each, multiple lines were added sequentially, with the number and position of lines varying randomly. An example of a synthetic image is shown in Fig 1(a).

Note that all synthetic lines used were straight lines. We conducted experiments explicitly using curved structures, however as there was no performance difference between training on curved or straight lines when detecting curved lines, it was decided that including curves was unnecessary.

\subsection{Real Mammogram Data}
We used real mammogram patches to illustrate qualitative results for curvilinear structure detection and to train and test spicule/non-spicule classification. Data were taken from a sequential set of 84 abnormal mammograms with biopsy-proven malignancy, drawn from a screening population (Nightingale Breast Centre, South Manchester University Hospitals Trust, UK), and from a set of 89 normal mammograms of the contralateral breasts of the same individuals (where disease was radiologically confirmed to be confined to one breast). All mammograms were digitised to a resolution of 90�m, using a Vidar CADPRO scanner. A 4x4 cm patch was extracted around each abnormality, and a similar patch was sampled randomly from each of the normal mammograms; examples are shown in \ref{f:} 4. For each abnormal patch an expert radiologist manually annotated some (though not necessarily all) of the spicules associated with the abnormality, resulting in a total of 555 spicule annotations. The expert spicule annotations for the abnormal images were used as a basis for selecting spicule pixels, though they were not sufficiently accurate to be used directly. To refine the annotations, we initialised a snake~\cite{Kass_etal_IJCV88} using each original annotation, and iterated it to convergence, using evidence from the linear structure probability image. We note that a similar technique has recently been published in detail by Muralidhar et al~\cite{Muralidhar_etal_TMI10}. As a result, the 555 refined spicule annotations identified a set of 36,514 spicule pixels. As outlined in section 4, we sampled the same number of pixels from the normal images, such that the detection probability distributions for the spicule and non-spicule samples were the same.

\subsection{Spicule Classification}
The four learning-based methods were also applied to the problem of spicule/non-spicule classification. Feature vectors were formed as above, and random forest classifiers were trained using balanced spicule/non-spicule training data, as outlined in section 4. To make effective use of the available data, we used a 10-fold cross-validation design. The set of normal and abnormal regions were divided into 10 groups so that the total number of normal and spicule pixels in each group were as close as possible to a 10th of the total and no two views from the same case were included in different groups. The samples in each group were then classified using a random forest trained on the samples from the remaining 9 groups. The classification results from each group were pooled to generate an unbiased class probability for each sampled pixel. These probabilities were used to compute an ROC curve for each training regime, and the area under the curve (Az) was computed and used as a measure of classification performance. The ROC curves and Az values for the three methods are shown in \ref{f:} 2 and \ref{t:} 2. These results demonstrate a clear advantage for DT-CWT/RF. As might be expected, because the Linop and Gaussian representations do not really capture profile shape, they perform significantly worse that the two representations that include phase.

In addition to computing a class vote for spicule membership at only those pixels in the selected training sets, the forests we have constructed can be used to compute results for whole region in each cross-fold group. Typical qualitative DT-CWT/RF results for a normal and abnormal region are shown in \ref{f:} 4. In the left column, the original regions are shown. The spiculations of the mass are clear and well defined, particularly to the south-east of the central mass. In the normal region, there are a set of structures that intersect in an approximate radial pattern that may trigger a feature detector erroneously. In the right column, the predicted spicule class membership is shown as hue varying from cyan (normal) to pink (spicule), modulated by the output of the DT-CWT/RF detection method. Note how the linear structures in the region of the mass are deemed highly likely to be spicules, whilst those in the normal region are not. This shows excellent promise as a means of providing a relevance measure to methods for abnormality detection.

\section{Discussion}
We have presented a discriminative learning-based approach to the detection and classification of curvilinear structures, based on a combination of DT-CWT representation of local structure and random forest classification. We have applied the method to the challenging problem of detecting and estimating the orientation of curvilinear structures in mammograms and distinguishing between normal and abnormal structures. The results of our experimental evaluation are extremely encouraging, and represent a significant improvement over the current state of the art. 

We have also introduced learning-based variants of three existing methods, demonstrating that whilst learning accounts for a significant part of this improvement, the choice of representation is also important and will have a different effect on performance depending on the task in hand. For example, constructing a representation based on the raw responses to Linop filters produces features that are excellent for estimating structure orientation but provide less information for determining structure shape and thus type. Conversely, features formed from the monogenic signal are good at determining structure type - most likely because of the inclusion of the phase measure - whilst they perform relatively poorly at detection and orientation estimation. For these reasons, it seems fair to conclude that the DT-CWT provides the best all round representation. It produced the strongest performance for all three tasks (curvilinear structure detection, orientation estimation and spicule classification). Moreover, as discussed in section 5.2, of all the methods, the DT-CWT incurs the least overhead when working with full-size real images that require block-wise classification/regression. For example, initial tests show that the structure detection and orientation regression can be performed on a full-size (~3000 x 2400 pixels) mammogram in ~1hr 30mins.

Our next goal is to show that improving the individual steps of curvilinear structure and orientation estimation result in a subsequent improvement for a high level task such as detecting patterns of spiculations indicative of disease. Moreover we hope to show that classification into structure type can further aid such tasks by focusing only (or at least more) on those structures most likely to be associated with disease.
