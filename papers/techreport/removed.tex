\comment{In retinography, segmenting blood vessels in digital fundus images can quantify venous beading and neovascularization, or mask out vessels when counting microaneurysms and exudates that indicate the presence and progression of diabetic retinopathy, the most common cause of blindness in the working population [ref]. (Note that as of 2002, no program existed to detect new vessel formation -- neovascularization --  in fundus images.)

Many of these applications are limited by their dependence on human operators to inspect images manually; a computerized system that could fulfil these functions automatically would make practical the widespread use of such screening, giving improved healthcare benefits.}

\comment{If you are one of the X\% of the population that suffers from diabetes, you have a Y\% chance of suffering from diabetic neuropathy and may go blind as a result. If the symptoms [not symptoms - rather what the doctor can see. warning signs perhaps] are detected early, however, there is a high [how high?] chance of treating the disease, leading to a higher quality of life. Detecting symptoms at an early stage requires widespread screening at a population level, which in turn requires high volume image inspection -- too high to do it manually. A computerized system that can analyze images automatically, detect early symptoms and suggest treatment would therefore be of significant benefit to population health.

There are N symptoms in which we are primarily interested: the presence of haemorrhages, microaneurysms and exudates in the retina; venous beading, where the diameter of a vessel oscillates along its length; and neovascularization, where new vessels appear where old vessels have died. The latter two require an accurate segmentation of the vessels in the retina, while the former would benefit from such a segmentation by eliminating many false positives in the detection. Other symptoms such as arteriolar narrowing and `nicking' may also be observed from the segmented vessel map and used to quantify other forms of retinopathy (such as hypertensive retinopathy).

There is, therefore, a significant body of work in segmenting vessels in fundus images of the retina...

It is important that we find the small vessels as well as the major ones, as this will allow us to detect neovascularization that is indicative of proliferative disease.

What are the challenges?
- to ensure that all vessels are picked up
  - is it sensible to enforce the condition that all vessels must be part of the tree? i.e. that any small vessels should exhibit some evidence of attachment to a larger vessel?
- 


***

Odd filters, such as the first derivative of a Gaussian, respond strongly at the edges of linear structure and have therefore been popular in some applications. The monogenic signal, used for ..., also falls into this category. In contrast, even filters such as the second derivative of a Gaussian tend to respond more strongly at the centre of linear structure and have found acceptance in other applications. 

Combining both odd and even filters (e.g. Gabor filters) gives a filter bank that not only responds strongly in the region of linear structure but can also allow fro small translations normal to the centreline through the use of phase information. Gabor filters are a popular choice, though they must be applied at multiple orientations to determine maximum response over orientation. An alternative is to use steerable filters that are efficient, or better still the dual tree complex wavelet transform (DT-CWT) [why is this better than steerable filters?].

The DTCWT has been applied to a number of fields including signal processing, face recognition and [something else]. Though it has shown some promise for medical image \emph{processing} (e.g. denoising), its use in medical image \emph{interpretation} has so far been limited to a handful of studies, most notably at the University of Manchester [IWDM,IPMI,MIUA]. Specifically, estimating orientation is nontrivial because an analytic formulation of orientation from the filter responses is not obvious, such that orientation must be regressed from the high-dimensional feature vector using machine learning techniques -- primarily Random Forests that act as a probabilistic nearest neighbour.
}

%In low-level processing, local orientation is used to steer computation, for example in non-maximal suppression for centre-line detection~\cite{Sonka_99}, profile extraction for structure classification~\cite{Zwiggelaar_etal_TMI04} and anisotropic diffusion~\cite{Perona_PAMI90}.  In intermediate-level processing, local orientation is used in grouping and tracking for extracting curvilinear features~\cite{Aylward_Bullitt_TMI02,Staal_etal_TMI04}.
